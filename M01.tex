% Options for packages loaded elsewhere
\PassOptionsToPackage{unicode}{hyperref}
\PassOptionsToPackage{hyphens}{url}
\PassOptionsToPackage{dvipsnames,svgnames,x11names}{xcolor}
%
\documentclass[
  11pt,
  letterpaper,
  DIV=11,
  numbers=noendperiod]{scrartcl}

\usepackage{amsmath,amssymb}
\usepackage{iftex}
\ifPDFTeX
  \usepackage[T1]{fontenc}
  \usepackage[utf8]{inputenc}
  \usepackage{textcomp} % provide euro and other symbols
\else % if luatex or xetex
  \usepackage{unicode-math}
  \defaultfontfeatures{Scale=MatchLowercase}
  \defaultfontfeatures[\rmfamily]{Ligatures=TeX,Scale=1}
\fi
\usepackage{lmodern}
\ifPDFTeX\else  
    % xetex/luatex font selection
\fi
% Use upquote if available, for straight quotes in verbatim environments
\IfFileExists{upquote.sty}{\usepackage{upquote}}{}
\IfFileExists{microtype.sty}{% use microtype if available
  \usepackage[]{microtype}
  \UseMicrotypeSet[protrusion]{basicmath} % disable protrusion for tt fonts
}{}
\makeatletter
\@ifundefined{KOMAClassName}{% if non-KOMA class
  \IfFileExists{parskip.sty}{%
    \usepackage{parskip}
  }{% else
    \setlength{\parindent}{0pt}
    \setlength{\parskip}{6pt plus 2pt minus 1pt}}
}{% if KOMA class
  \KOMAoptions{parskip=half}}
\makeatother
\usepackage{xcolor}
\usepackage{svg}
\setlength{\emergencystretch}{3em} % prevent overfull lines
\setcounter{secnumdepth}{5}
% Make \paragraph and \subparagraph free-standing
\ifx\paragraph\undefined\else
  \let\oldparagraph\paragraph
  \renewcommand{\paragraph}[1]{\oldparagraph{#1}\mbox{}}
\fi
\ifx\subparagraph\undefined\else
  \let\oldsubparagraph\subparagraph
  \renewcommand{\subparagraph}[1]{\oldsubparagraph{#1}\mbox{}}
\fi


\providecommand{\tightlist}{%
  \setlength{\itemsep}{0pt}\setlength{\parskip}{0pt}}\usepackage{longtable,booktabs,array}
\usepackage{calc} % for calculating minipage widths
% Correct order of tables after \paragraph or \subparagraph
\usepackage{etoolbox}
\makeatletter
\patchcmd\longtable{\par}{\if@noskipsec\mbox{}\fi\par}{}{}
\makeatother
% Allow footnotes in longtable head/foot
\IfFileExists{footnotehyper.sty}{\usepackage{footnotehyper}}{\usepackage{footnote}}
\makesavenoteenv{longtable}
\usepackage{graphicx}
\makeatletter
\def\maxwidth{\ifdim\Gin@nat@width>\linewidth\linewidth\else\Gin@nat@width\fi}
\def\maxheight{\ifdim\Gin@nat@height>\textheight\textheight\else\Gin@nat@height\fi}
\makeatother
% Scale images if necessary, so that they will not overflow the page
% margins by default, and it is still possible to overwrite the defaults
% using explicit options in \includegraphics[width, height, ...]{}
\setkeys{Gin}{width=\maxwidth,height=\maxheight,keepaspectratio}
% Set default figure placement to htbp
\makeatletter
\def\fps@figure{htbp}
\makeatother

\usepackage{svg}
\KOMAoption{captions}{tableheading}
\makeatletter
\@ifpackageloaded{caption}{}{\usepackage{caption}}
\AtBeginDocument{%
\ifdefined\contentsname
  \renewcommand*\contentsname{Table of contents}
\else
  \newcommand\contentsname{Table of contents}
\fi
\ifdefined\listfigurename
  \renewcommand*\listfigurename{List of Figures}
\else
  \newcommand\listfigurename{List of Figures}
\fi
\ifdefined\listtablename
  \renewcommand*\listtablename{List of Tables}
\else
  \newcommand\listtablename{List of Tables}
\fi
\ifdefined\figurename
  \renewcommand*\figurename{Figure}
\else
  \newcommand\figurename{Figure}
\fi
\ifdefined\tablename
  \renewcommand*\tablename{Table}
\else
  \newcommand\tablename{Table}
\fi
}
\@ifpackageloaded{float}{}{\usepackage{float}}
\floatstyle{ruled}
\@ifundefined{c@chapter}{\newfloat{codelisting}{h}{lop}}{\newfloat{codelisting}{h}{lop}[chapter]}
\floatname{codelisting}{Listing}
\newcommand*\listoflistings{\listof{codelisting}{List of Listings}}
\makeatother
\makeatletter
\makeatother
\makeatletter
\@ifpackageloaded{caption}{}{\usepackage{caption}}
\@ifpackageloaded{subcaption}{}{\usepackage{subcaption}}
\makeatother
\ifLuaTeX
  \usepackage{selnolig}  % disable illegal ligatures
\fi
\usepackage{bookmark}

\IfFileExists{xurl.sty}{\usepackage{xurl}}{} % add URL line breaks if available
\urlstyle{same} % disable monospaced font for URLs
\hypersetup{
  pdftitle={EPPS Math and Coding Camp},
  pdfauthor={Karl Ho},
  colorlinks=true,
  linkcolor={blue},
  filecolor={Maroon},
  citecolor={Blue},
  urlcolor={Blue},
  pdfcreator={LaTeX via pandoc}}

\title{EPPS Math and Coding Camp}
\usepackage{etoolbox}
\makeatletter
\providecommand{\subtitle}[1]{% add subtitle to \maketitle
  \apptocmd{\@title}{\par {\large #1 \par}}{}{}
}
\makeatother
\subtitle{Day 1: Fundamentals}
\author{Karl Ho}
\date{}

\begin{document}
\maketitle

\section{Introduction to
Fundamentals}\label{introduction-to-fundamentals}

\begin{itemize}
\tightlist
\item
  Variables and Constants
\end{itemize}

\begin{itemize}
\tightlist
\item
  Sets
\end{itemize}

\begin{itemize}
\tightlist
\item
  Operators
\end{itemize}

\begin{itemize}
\tightlist
\item
  Relations
\end{itemize}

\begin{itemize}
\tightlist
\item
  Level of Measurement
\end{itemize}

\begin{itemize}
\tightlist
\item
  Notation
\end{itemize}

\begin{itemize}
\tightlist
\item
  Proofs
\end{itemize}

\section{Variables and Constants}\label{variables-and-constants}

\begin{itemize}
\tightlist
\item
  Definition and Examples
\end{itemize}

\begin{itemize}
\tightlist
\item
  Applications in Research
\end{itemize}

\section{Sets}\label{sets}

\begin{itemize}
\tightlist
\item
  Definition and Types
\end{itemize}

\begin{itemize}
\tightlist
\item
  Set Operations
\end{itemize}

\begin{itemize}
\tightlist
\item
  Venn Diagrams
\end{itemize}

\section{Operators}\label{operators}

\begin{itemize}
\tightlist
\item
  Basic Operators: +, -, *, /
\end{itemize}

\begin{itemize}
\tightlist
\item
  Advanced Operators: Summation, Product
\end{itemize}

\begin{itemize}
\tightlist
\item
  Set Operators: Union, Intersection, Difference
\end{itemize}

\section{Relations}\label{relations}

\begin{itemize}
\tightlist
\item
  Binary Relations
\end{itemize}

\begin{itemize}
\tightlist
\item
  Functions as Relations
\end{itemize}

\section{Level of Measurement}\label{level-of-measurement}

\begin{itemize}
\tightlist
\item
  Nominal, Ordinal, Interval, Ratio
\end{itemize}

\begin{itemize}
\tightlist
\item
  Examples and Applications
\end{itemize}

\section{Notation}\label{notation}

\begin{itemize}
\tightlist
\item
  Common Symbols and Their Meanings
\end{itemize}

\begin{itemize}
\tightlist
\item
  How to Read Mathematical Expressions
\end{itemize}

\section{Operators}\label{operators-1}

\subsection{Summation and Multiplication
Notations}\label{summation-and-multiplication-notations}

\textbf{Summation Notation} \[ \sum_{i=1}^{3} x_i = x_1 + x_2 + x_3
\]

\textbf{Multiplication Notation} \[
\prod_{i=1}^{3} x_i = x_1 \times x_2 \times x_3
\]

\begin{center}\rule{0.5\linewidth}{0.5pt}\end{center}

\section{Examples}\label{examples}

\subsection{Summation and Multiplication in
Practice}\label{summation-and-multiplication-in-practice}

Consider the following sets of values for \(x_i\):

\begin{itemize}
\tightlist
\item
  \(x_1 = 2\)
\item
  \(x_2 = 4\)
\item
  \(x_3 = 6\)
\end{itemize}

\textbf{Summation:} \[ \sum_{i=1}^{3} x_i = 2 + 4 + 6 = 12 \]

\textbf{Multiplication:}
\[ \prod_{i=1}^{3} x_i = 2 \times 4 \times 6 = 48 \]

\section{Graphical Illustration}\label{graphical-illustration}

\subsection{Venn Diagrams for Set
Operations}\label{venn-diagrams-for-set-operations}

\includesvg{M01_files/mediabag/Venn0001.svg}

\section{Proofs}\label{proofs}

\begin{itemize}
\tightlist
\item
  Direct and Indirect Proofs
\end{itemize}

\begin{itemize}
\tightlist
\item
  Example Proofs
\end{itemize}

\section{Exercises}\label{exercises}

\begin{itemize}
\tightlist
\item
  Variables and Constants Exercise
\end{itemize}

\begin{itemize}
\tightlist
\item
  Sets and Set Operations Exercise
\end{itemize}

\begin{itemize}
\tightlist
\item
  Operators and Relations Exercise
\end{itemize}

\section{Graphical Illustrations}\label{graphical-illustrations}

\begin{itemize}
\tightlist
\item
  Venn Diagrams
\end{itemize}

\begin{itemize}
\tightlist
\item
  Graphs of Functions and Relations
\end{itemize}

\section{Suggested Readings and
Videos}\label{suggested-readings-and-videos}

\begin{itemize}
\tightlist
\item
  Read Moore and Siegel, Chapter 1
\end{itemize}

\begin{itemize}
\tightlist
\item
  Video:
  \href{https://www.youtube.com/watch?v=OCNXS_m1HWU&list=PLwPDkKEXCNflNrtW4uG2mcOY1Q0ByREuP}{Bringing
  the set operations together}
\item
  Three Blue One Brown
  \href{https://www.youtube.com/watch?v=kjBOesZCoqc&list=PL0-GT3co4r2y2YErbmuJw2L5tW4Ew2O5B}{Essence
  of linear algebra series}
\end{itemize}

\section{Q\&A}\label{qa}

\begin{itemize}
\tightlist
\item
  Open Discussion
\end{itemize}

\begin{itemize}
\tightlist
\item
  Address Questions and Clarifications
\end{itemize}

\section{Exercise 1: Properties of
Sets}\label{exercise-1-properties-of-sets}

\subsection{Question 5}\label{question-5}

Consider the sets \(A = \{1, 5, 10\}\) and \(B = \{1, 2, \ldots, 10\}\).

\begin{enumerate}
\def\labelenumi{\arabic{enumi}.}
\tightlist
\item
  Is \(A \subset B\), \(B \subset A\), both, or neither?
\item
  What is \(A \cup B\)?
\item
  What is \(A \cap B\)?
\item
  Partition \(B\) into two sets, \(A\) and everything else. Call
  everything else \(C\). What is \(C\)?
\item
  What is \(A \cup C\)?
\item
  What is \(A \cap C\)?
\end{enumerate}

\section{Exercise 1: Properties of
Sets}\label{exercise-1-properties-of-sets-1}

\subsection{Solution}\label{solution}

\begin{enumerate}
\def\labelenumi{\arabic{enumi}.}
\tightlist
\item
  \(A \subset B\)
\item
  \(A \cup B = \{1, 2, 3, 4, 5, 6, 7, 8, 9, 10\}\)
\item
  \(A \cap B = \{1, 5, 10\}\)
\item
  \(C = B \setminus A = \{2, 3, 4, 6, 7, 8, 9\}\)
\item
  \(A \cup C = \{1, 2, 3, 4, 5, 6, 7, 8, 9, 10\}\)
\item
  \(A \cap C = \emptyset\) :::
\end{enumerate}

\section{Exercise 2: Cartesian
Product}\label{exercise-2-cartesian-product}

\subsection{Question 6}\label{question-6}

Write an element of the Cartesian product \([0, 1] \times (1, 2)\).

\section{Exercise 2: Cartesian
Product}\label{exercise-2-cartesian-product-1}

\subsection{Solution}\label{solution-1}

An element of the Cartesian product \([0, 1] \times (1, 2)\) is
\((x, y)\) where \(x \in [0, 1]\) and \(y \in (1, 2)\).

Example: \((0.5, 1.5)\) :::

\section{Exercise 3: Irrational
Numbers}\label{exercise-3-irrational-numbers}

\subsection{Question 7}\label{question-7}

Prove that \(\sqrt{2}\) is an irrational number. That is, show that it
cannot be written as the ratio of two integers, \(p\) and \(q\).

\section{Exercise 3: Irrational
Numbers}\label{exercise-3-irrational-numbers-1}

\subsection{Solution}\label{solution-2}

Assume \(\sqrt{2}\) is rational, meaning it can be written as
\(\frac{p}{q}\) where \(p\) and \(q\) are coprime integers (no common
factors other than 1).

\[
\sqrt{2} = \frac{p}{q} \implies 2 = \frac{p^2}{q^2} \implies 2q^2 = p^2
\]

This implies \(p^2\) is even, so \(p\) must be even. Let \(p = 2k\) for
some integer \(k\).

\[
2q^2 = (2k)^2 \implies 2q^2 = 4k^2 \implies q^2 = 2k^2
\]

This implies \(q^2\) is even, so \(q\) must be even. But if \(p\) and
\(q\) are both even, they have a common factor of 2, contradicting the
assumption that they are coprime.

Therefore, \(\sqrt{2}\) is irrational. :::

\section{Exercise 4: Even and Odd
Numbers}\label{exercise-4-even-and-odd-numbers}

\subsection{Question 8}\label{question-8}

Prove that: 1. The sum of any two even numbers is even. 2. The sum of
any two odd numbers is even. 3. The sum of any odd number with any even
number is odd.

\section{Exercise 4: Even and Odd
Numbers}\label{exercise-4-even-and-odd-numbers-1}

\subsection{Solution}\label{solution-3}

\begin{enumerate}
\def\labelenumi{\arabic{enumi}.}
\item
  Let \(a\) and \(b\) be even numbers. Then \(a = 2m\) and \(b = 2n\)
  for some integers \(m\) and \(n\). \[
  a + b = 2m + 2n = 2(m + n)
  \] Since \(m + n\) is an integer, \(a + b\) is even.
\item
  Let \(a\) and \(b\) be odd numbers. Then \(a = 2m + 1\) and
  \(b = 2n + 1\) for some integers \(m\) and \(n\). \[
  a + b = (2m + 1) + (2n + 1) = 2m + 2n + 2 = 2(m + n + 1)
  \] Since \(m + n + 1\) is an integer, \(a + b\) is even.
\item
  Let \(a\) be an odd number and \(b\) be an even number. Then
  \(a = 2m + 1\) and \(b = 2n\) for some integers \(m\) and \(n\). \[
  a + b = (2m + 1) + 2n = 2m + 2n + 1 = 2(m + n) + 1
  \] Since \(m + n\) is an integer, \(a + b\) is odd. :::
\end{enumerate}



\end{document}
